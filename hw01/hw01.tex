\documentclass[letterpaper, 11pt]{article}
%\usepackage[hmargin = 1in, vmargin = 1in]{geometry}
\usepackage{amsmath}
\usepackage{amssymb}
\usepackage{amsthm}
\usepackage{enumitem}
\usepackage{mathrsfs}
\usepackage{tikz}
\usepackage{graphicx}
\usepackage{hyperref}
\usepackage{algorithmicx}
\usepackage{algpseudocode}
\setlength{\headheight}{14pt}
\usepackage{fancyhdr}
\pagestyle{fancy}
\rhead{Gabriel Wallace}
\lhead{Comp Sci 5130}

\hypersetup{
	colorlinks=true,
	linkcolor=blue,
	urlcolor=blue
}

\newcommand{\card}{\text{Card}}
\newcommand{\N}{\mathbb{N}}
\newcommand{\R}{\mathbb{R}}
\newcommand{\Z}{\mathbb{Z}}
\newcommand{\Q}{\mathbb{Q}}

\newcommand{\inv}{^{-1}}
\newcommand{\abs}[1]{\lvert #1 \rvert}
\newcommand{\hwnumber}[1]{\medskip \noindent\textbf{#1.} \smallskip}
\newcommand{\Mod}[1]{\ \mathrm{mod}\ #1}
\newcommand{\Alg}[1]{\medskip \noindent\textbf{ALGORITHM} \( #1 \)} 
\newcommand{\To}{\textbf{ to }}

\begin{document}
\begin{center}
	{\LARGE Homework 1}\\
\end{center}

The text of the questions to this homework can be found
\href{https://github.com/badriadhikari/Algorithms-2020fall/blob/master/homeworks/Homework01.md}{here}.

\hwnumber{1} The functions can be ranked as follows:
\[O(1) < O(\lg n) = O(k \lg n) < O(n) = O(2n) = O(kn)\ = O(100000 n) \]
\[ < O(n\lg n) < O(n^2) < O(n^{100000}) < O(n!)\]

\hwnumber{2} 

This statement is true, but not very useful and hence meaningless.
Big O notation gives us the upper bound of a given function, but it doesn't tell
us the actual growth of a given function. Saying that the running time of
algorithm $A$ is at least as big as an upper bound is not very helpful. For
example, let $f(n) = 0$, then $f(n) = O(n^2)$. Since the running time of
algorithms is always non-negative, then $T(n)$ grows at least as fast as $f(n)$,
which is true for every function. Thus, this statement has told us nothing of
value about algorithm $A$, and is therefore meaningless.  

\hwnumber{3}
The first one is true, but the second one is false. 

\begin{proof}[Proof of 1] 
Let $c = 2$ and $n_0 = 1$, then 
\[0 \leq 2^{n+1} = 2 \cdot 2^n \leq 2\cdot 2^n = c \cdot 2^n\]
for all $n \geq n_0 = 1$. So the definition of big O is satisfied. 
\end{proof}

\begin{proof}[Proof of 2]
Assume that $2^{2n} = O(2^n)$. The there exists constants $c$ and $n_0$ such that 
\[0 \leq 2^{2n} = 2^n \cdot 2^n \leq c \cdot 2^n\]
for all $n \geq n_0$. Thus, $c \geq 2^n$, for an arbitrarily large value of $n$,
a contradiction since $c$ is a constant. Thus, $2^{2n} \not= O(2^n)$. 
\end{proof}

\newpage
\hwnumber{4}

One way we can compare these functions is by plotting them. We use the Python
package \texttt{matplotlib} (example code
\href{https://raw.githubusercontent.com/gwallace04/cs5130/master/plot.py}{here})
to plot the functions as in \ref{fig:plot}. Since it is easy for the plots to
become cluttered, we compare two or three functions at a time. 

\begin{figure}[h]
\includegraphics[width=\textwidth]{functions.png}
\caption{Comparing functions}
\label{fig:plot}
\end{figure}

Continuing on in this manner, we get the following list ordered fastest growing
to slowest:

\begin{enumerate}
\item $2^{2^{n+1}}$
\item $2^{2^n}$
\item $(n + 1)!$
\item $n!$
\item $e^n$
\item $n \cdot 2^n$
\item $2^n$
\item $n^{\lg \lg n}$
\item $(\lg n)!$
\item $4^{\lg n} = n^2$
\item $\lg(n!) = n \lg n$
\item $(\sqrt{2})^{\lg n}$
\item $n$
\item $\lg^2 n$
\item $\ln n$
\item $\sqrt{\lg n}$
\item $n^{1 / \lg n} = 1$
\end{enumerate}



\end{document}
